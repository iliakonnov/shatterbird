% Языки, которые могут быть использованы.
\usepackage[english,russian]{babel}

% FreeSerif — альтернатива TimesNewRoman
\usepackage{fontspec}
\defaultfontfeatures{Ligatures={TeX}} 
\setmainfont[Ligatures={TeX}]{FreeSerif}

% Используем Fira Mono для моноширинного текста
\usepackage[scaled=0.8]{FiraMono}

% Чтобы сделать буковки ещё красивее.
\usepackage{microtype}

% Картинки, вращение и всё такое. Причём даже в режиме черновика
\usepackage[final]{graphicx}
\usepackage{float}
\usepackage[draft=no]{svg}
\usepackage{pdfpages}

% Чтобы легко менять стиль списков
\usepackage[shortlabels]{enumitem}

\usepackage{unicode-math}
\usepackage{csquotes}
\usepackage{adjustbox}
\usepackage{relsize}

% https://tex.stackexchange.com/questions/14967/source-code-listing-with-frame-around-code
\usepackage{listings}
\lstset{frame=lrbt,xleftmargin=\fboxsep,xrightmargin=-\fboxsep}
\lstset{basicstyle=\ttfamily,breaklines=true}

% https://qna.habr.com/answer?answer_id=1140867
\bibliographystyle{gost2008}
\usepackage[
	style=gost-numeric, % стиль цитирования и библиографии, см. документацию biblatex-gost
	language=auto,  % использовать язык из babel
	autolang=other, % многоязычная библиография
	parentracker=true,
	backend=biber,
	hyperref=true,
	bibencoding=utf8,
	sorting=ntvy,  % сортировка: имя, заголовок, том, год
]{biblatex}
\addbibresource{bibliography.bib}
\DeclareSourcemap{
    \maps{
        \map{% если @online, то устанавливаем media=eresource.
            \step[typesource=online, fieldset=media, fieldvalue=eresource]
        }
    }
}

% Отсупы как по госту.
\usepackage[
	includeheadfoot,
	left=20mm,
	right=10mm,
	top=0mm,
	headheight=25mm,
	footskip=5\baselineskip, % footskip is the distance between the textbody and the baseline of the footer
	bottom=15mm,
]{geometry}

\savegeometry{original}
\geometry{bottom=10mm}
\savegeometry{nofooter}
\loadgeometry{original}

%% Некоторые переменные, которые появляются более одного раза

% Название курсовой. Не забывать проценты в конце строк.
\newcommand{\docTitle}{%
	Сервис для индексирования и просмотра репозиториев%
}

\newcommand{\docTitleEng}{%
	Repository Indexing and Viewing Service%
}

% Год написания курсовой.
\newcommand{\YEAR}{
	\the\year{}
}

% RU — код страны
% 17701729 — код НИУ ВШЭ
% 05.15 — регистрационный номер (https://docs.cntd.ru/document/566085681)
% 		В данном случае, Прикладное ПО / Информационные системы для решения специфических отраслевых задач
% 01 — номер редакции документа
% 33 — код вида документа (https://www.swrit.ru/doc/espd/19.101-77.pdf#page=4)
% 01 — номер документа данного вида
%  1 — номер части документа
\newcommand{\docId}{RU.17701729.03.10-01 12 01-1}


% Footer and header
\usepackage{fancyhdr}
\pagestyle{fancy}
\fancyhf{}
\chead{
	%\bf % Сделать жирненьким
	\thepage\\
	\docId
}
\fancyhead[RO]{%
    % В правый верхний колотитул пихаю номер приложения
    {\ifnum\value{addendum}>0 ПРИЛОЖЕНИЕ \theaddendum \fi}
}
\cfoot{%
	\adjustbox{valign=b}{ % baseline of footer at bottom, so margins are correct
		\begin{tabular}{| l | l | l | l | l |}
			\hline & & & & \\
			\hline Изм. & Лист & № докум. & Подп. & Дата \\
			\hline \docId &  &  &  &  \\
			\hline Инв. № подл. & Подп. и дата & Взам. инв. № & Инв. № дубл. & Подп. и дата № \\
			\hline
		\end{tabular}
	}
}
\fancypagestyle{nofooter}{%
	\fancyfoot{}%
}

\renewcommand{\headrulewidth}{0pt}
\renewcommand{\footrulewidth}{0pt}

% Повернуть на 90 градусов
\newcommand{\rot}[2]{\rotatebox[origin=c]{90}{\enspace\parbox{#1 - 0.5em}{#2}}}

% Повторить #1 раз текст #2: \Repeat{#1}{#2}
\usepackage{expl3}
\ExplSyntaxOn
\cs_new_eq:NN \Repeat \prg_replicate:nn
\ExplSyntaxOff

% Продвинутые таблицы
\usepackage{tabularx}
\usepackage{multirow}
\usepackage{xltabular}
% И сразу делаем чтобы колонки X центрировались по вертикали
\def\tabularxcolumn#1{m{#1}}
\newcolumntype{Y}{>{\centering\arraybackslash}X}

% Подсчет числа страниц
\usepackage{lastpage}

% Расчет всяких размеров
\usepackage{calc}

% Названия разделов по центру (но titlesec конфликтует с hyperref)
\usepackage{titlesec}
\titleformat{\section}
  {\centering\Large\bfseries}
  {\thesection}
  {.5em}
  {\MakeUppercase}
% И с точкой в конце.
\titlelabel{\thetitle.\quad}

% После названия раздела надо делать отступы у абзацев.
\usepackage{indentfirst}

% https://tex.stackexchange.com/a/347803
\usepackage{varwidth}

\newcommand{\placename}{
	\underline{\hspace{4cm}}
}
\newcommand{\placedate}{
	«\underline{\hspace{1em}}» \underline{\hspace{3cm}} \YEAR г.
}

\newcounter{addendum}
\makeatletter
\newcommand{\addendum}[1]{
    \stepcounter{addendum}

	% Приложения есть в оглавлении
	\phantomsection
	\addcontentsline{toc}{section}{Приложение \arabic{addendum}: #1}

	\section*{#1}
	
	% Сбросить все счетчики
	\setcounter{equation}{0}
    \setcounter{figure}{0}
    \setcounter{table}{0}

	% Треш и угар с макросами, чтобы обойти проблемы между titlsec и hyperref
	% Цель: \NR@gettitle{Приложение 1: …}, но нужно **сначала** раскрыть \arabic{addendum}
	\edef\addendumTitle{\noexpand\NR@gettitle{Приложение \arabic{addendum}: #1}}
	\addendumTitle
}
\makeatother

% Ещё больше подсекций
\newcommand{\subsubsubsection}[1]{\paragraph{#1}\mbox{}\\}
\setcounter{secnumdepth}{4}
\setcounter{tocdepth}{4}

% TODOшечки
\usepackage{ifdraft}
\ifdraft{
	% Если \document[draft], то нам нужно место в полях для todo.
	\geometry{left=40mm, right=5mm, marginparwidth=35mm}
	\reversemarginpar  % причём слева они выглядят симпатичнее
	\savegeometry{original}
}{}
\usepackage[obeyDraft,textsize=footnotesize,textwidth=35mm]{todonotes}

% Добавляем гипертекстовое оглавление в PDF
% hyperref должен быть последним
\usepackage[hidelinks]{hyperref}