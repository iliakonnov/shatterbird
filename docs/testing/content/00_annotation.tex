\section*{АННОТАЦИЯ}

Программа и методика испытаний – это документ, в котором содержится информация о
программном продукте, а также полное описание приемочных испытаний для данного
программного продукта.

Настоящая Программа и методика испытаний для «Конструктора схем разделения секретов»
содержит следующие разделы: «Объект испытаний», «Цель испытаний», «Требования к
программе», «Требования к программной документации», «Средства и порядок испытаний»,
«Методы испытаний».

В разделе «Объект испытаний» указаны наименование, область применения и обозначение
испытуемой программы.

В разделе «Цель испытаний» указана цель проведения испытаний.

В разделе «Требования к программе» указаны требования, подлежащие проверке во время
испытаний и заданные в техническом задании на проверку.

В разделе «Требования к программным документам» указаны состав программной
документации, предъявляемой на испытания, а также специальные требования,

В разделе «Средства и порядок испытаний» указаны технические и программные средства,
используемые во время испытаний, а также порядок проведения испытаний.

В разделе «Методы испытаний» приведены описания используемых методов испытаний.

Настоящий документ разработан в соответствии с требованиями:
\begin{enumerate}[1)]
	\item ГОСТ 19.101-77 Виды программ и программных документов \cite{gost:19.101-77};
	\item ГОСТ 19.102-77 Стадии разработки \cite{gost:19.102-77};
	\item ГОСТ 19.103-77 Обозначения программ и программных документов \cite{gost:19.103-77};
	\item ГОСТ 19.104-78 Основные надписи \cite{gost:19.104-78};
	\item ГОСТ 19.105-78 Общие требования к программным документам \cite{gost:19.105-78};
	\item ГОСТ 19.106-78 Требования к программным документам, выполненным печатным способом \cite{gost:19.106-78};
	\item ГОСТ 19.301-79 Программа и методика испытаний. Требования к содержанию и оформлению \cite{gost:19.301-79};
	\item Изменения к данному документу оформляются согласно ГОСТ 19.603-78 \cite{gost:19.603-78}, ГОСТ 19.604-78 \cite{gost:19.604-78}.
\end{enumerate}

\clearpage