\section{ТРЕБОВАНИЯ К ПРОГРАММНОЙ ДОКУМЕНТАЦИИ}

\subsection{Состав программной документации}

\begin{enumerate}
    \item «\docTitle». Техническое задание (ГОСТ 19.201-78 \cite{gost:19.201-78});
    \item «\docTitle». Программа и методика испытаний (ГОСТ 19.301-79 \cite{gost:19.301-79});
    \item «\docTitle». Руководство оператора (ГОСТ 19.505-79  \cite{gost:19.505-79}).
    \item «\docTitle». Текст программы (ГОСТ 19.401-79 \cite{gost:19.401-79}).
    \item Текст выпускной квалификационной работы на тему «\docTitle».
\end{enumerate}

\subsection{Специальные требования к программной документации}

\begin{enumerate}
    \item Все документы к программе должны быть выполнены и оформлены в соответствии с
    ГОСТ 19.101-77, ГОСТ 19.103-77, ГОСТ 19.104-78, ГОСТ 19.105-78, с ГОСТ 19.106-78 и ГОСТ к этому виду документа.
    \item Текст выпускной квалификационной работы должен быть загружена в систему Антиплагиат через LMS «НИУ ВШЭ».
    \item Техническое задание и текст ВКР, титульные листы других документов должны быть подписаны руководителем разработки и исполнителем.
    \item Документация и программа сдаются в электронном виде в формате .pdf или .docx.
    \item Не позднее, чем за 4 календарных дня до утвержденной даты защиты ВКР все материалы выпускной работы: программная документация, программный проект, отзыв руководителя, отчет системы Антиплагиат — должны быть загружены в проект дисциплины в личном кабинете в информационной образовательной среде SmartLMS НИУ ВШЭ.
\end{enumerate}

\clearpage