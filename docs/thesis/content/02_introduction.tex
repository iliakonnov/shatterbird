\addsec{Введение}

При разработке программного обеспечения часто необходимо читать исходный код проектов, причем разработчик может быть первоначально не знаком с изучаемым исходным кодом. Это часто случается при изучении поведения внешних систем или при недостатке документации. При этом для облегчения понимания кода разработчику помогают специальные инструменты, которые способны анализировать код и помогать в изучении проекта.

Такие инструменты должны предоставлять следующий функционал:
\begin{itemize}
    \item Отображение подсказки при наведении курсора на \gls{symbol} в исходном коде: информация о типе переменной, документация.
    \item Переход к определению символа. Это может быть переход к определению функции от места её использования, что часто бывает полезно для изучения процесса обработки данных в программе. Это также может быть переход к определению типа от места его использования, что позволит увидеть содержащиеся в нём поля.
    \item Поиск всех использований символа — обратный функционал к операции перехода к определению символа. Например, позволяет найти все места, где вызывается функция, и исходя из этого лучше понять её входные параметры.
\end{itemize}

Обычно для этого применяются \gls{IDE}, которые требуют скачивания проекта на компьютер разработчика и затем достаточно длительной загрузки всех зависимостей, компиляции и индексации. До завершения этих операций IDE не способна проанализировать и предложить разработчику подсказки. Это значительно замедляет изучение сложных систем, состоящих из многих отдельных сервисов.

Однако существует возможность значительно уменьшить время от необходимости изучать код проекта до получения подсказок, если заранее и централизованно индексировать \gls{repo}, ещё до того как данные индексации понадобятся разработчикам. Можно выполнять компиляцию и индексацию на каждом новом изменении в репозитории, чтобы, при необходимости, актуальные данные всегда были доступны.

Такой подход позволит обеспечить разработчикам быстрый и удобный доступ к функционалу навигации по коду без необходимости выполнения длительной и ресурсоёмкой индексации. Однако создание такой системы требует решения многих задач: хранения данных репозитория и индексов, реализация пользовательского интерфейса, предоставление частичных ответов в условии отсутствия актуальных индексов по тем или иным причинам.

Подобные системы разработаны во многих больших компаниях, занимающихся разработкой программного обеспечения (Google, Facebook\facebook, Yandex), но обычно такие системы доступны только их сотрудников и очень тесно интегрированы с проприетарными системами сборки. Также существует несколько открытых аналогов и коммерческие системы, но те обладают своими недостатками. В этой работе приводятся методы решения задач индексации, а также приводится описание реализованной системы индексирования Git-репозитория.

Выпускная квалификационная работа состоит из введения, трех глав и заключения:

\begin{itemize}
    \item В \hyperref[chap:overview]{\textbf{первой главе}} дается постановка проблемы и анализируются возможные пути её решения.
    \item Во \hyperref[chap:solution]{\textbf{второй главе}} описываются основные идеи и алгоритмы разработанной системы.
    \item В \hyperref[chap:architecture]{\textbf{третьей главе}} рассматривается архитектура системы и описываются детали её реализации.
    \item В \hyperref[chap:final]{\textbf{заключении}} резюмируется функциональный и технический подход к разработке системы.
\end{itemize}

\clearpage
