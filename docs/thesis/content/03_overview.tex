\chapter{Обзор предметной области и анализ существующих решений}
\label{chap:overview}

\section{Предметная область}

При разработке программного обеспечения часто необходимо читать код смежных проектов, чтобы уточнить детали поведения, прояснить особенности API, или найти существующую реализацию определенной функциональности.
Также часто используется система хранения версий Git \cite{git-paper}, а также какая-либо платформа для организации совместной работы над проектами (например, GitLab, BitBucket).
Эти платформы предоставляют возможность с использованием веб-интерфейса читать код проектов, но не предоставляют возможностей по навигации, сравнимых с возможностями полновесной \gls{IDE}.
Это сильно усложняет процесс погружения в незнакомые проекты, потому что разработчик лишается привычных подсказок (например, вывод типов, документация) и возможностей по навигации (например, go to definition).
Обычно эти возможности доступны только при локальном выкачивании проекта и последующем его индексировании с помощью \gls{IDE}, однако это занимает достаточно много времени.
Особенно ситуация усугубляется при использовании микросервисной архитектуры, когда количество репозиториев становится большим.

Таким образом, у разработчиков существует потребность в решении, которые бы предоставляло им продвинутые возможности навигации и поиска по коду, но без необходимости трудоемкого индексирования.
Индексация кода должна осуществляется автоматически и асинхронно, в то время как разработчики должны иметь возможность использовать ранее построенные индексы с использованием специального приложения.

Особенно остро эта проблема стоит в связи с нынешней проблемой в продлении лицензий на многие коммерческие продукты.

\section{Существующие решения}

Для индексации кода существует множество различных существующих решений, рассмотрим некоторые из них:

\begin{itemize}
    \item SourceBrowser \cite{SourceBrowser} для .NET — позволяет сгенерировать для проекта статический сайт с возможностью перейти к определению любого \glslink{symbol}{символа} и возможностью найти все использования того или иного символа. Однако, поддерживает только языки на базе .NET.
    
    \item Google CodeSearch \cite{rsc-regexp4} — осуществляет быстрый regexp-поиск по кодовой базе, но не осуществляет никакого семантического анализа. На данный момент не поддерживается.
    
    \item The Elixir Cross Referencer \cite{elixir-crossrefrencer} — отлично индексирует код на C/C++, но не поддерживает другие языки.
    
    \item Glean \cite{facebook-Glean} — система, решающая задачу хранения данных индексации и выполнения запросов к ним. Но она не предоставляет удобного веб-интерфейса, а также до сих пор находится в разработке.
    
    \item SourceGraph \cite{miljenovic2010sourcegraph} — эффективно решает все индексации и навигации проблемы, но на данный момент больше не является открытым и бесплатным решением.
\end{itemize}

Все эти решения либо не являются открытыми и бесплатными, либо поддерживают только очень ограниченное количество языков программирования, либо не предоставляют необходимый функционал.

\section{Требования к системе}

Таким образом, можно сформулировать основные требования к системе:

\begin{enumerate}
    \item Система должна иметь открытый исходный код и распространяться под свободной лицензией;
    \item Система должна иметь возможность индексировать код на различных языках программирования. Обязательно должны поддерживаться языки программирования Rust и TypeScript;
    \item Система должна предоставлять функционал поиска определения символа, поиска всех использований символа и отображение справочной информации при наведении курсора на символ;
    \item Система должна иметь веб-интерфейс, в котором пользователь сможет просматривать содержимое репозитория;
    \item Система должна быть простой в развертывании и использовании;
    \item Система должна иметь возможность ограниченно использовать индексы построенные с использованием предыдущей версии репозитория даже при просмотре той версии, которая ещё не была проиндексирована.
\end{enumerate}

\section{Выводы по главе}

С одной стороны, система решает действительно важную и актуальную проблему разработчиков. Она может позволить заметно облегчить их работу и уменьшить время решения некоторых задач. С другой стороны, хотя аналоги и существуют, но на данный момент не было найдено аналогичных решений, распространяемых под свободной лицензией, поэтому разработка системы весьма актуальна. Также были сформулированы основные требования к этой системе.

\clearpage