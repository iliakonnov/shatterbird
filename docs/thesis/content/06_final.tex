\addchap{Заключение}
\label{chap:final}

Благодаря данной выпускной квалификационной работе, была разработана открытая и бесплатная система, облегчающая работу разработчиков программного обеспечения. Такая система позволяет им более продуктивно изучать исходный код незнакомых проектов и быстрее находить в нём ответы на свои вопросы. Причем система является расширяемой и может поддерживать многие языки программирования, а также проста в развертывании и использовании. 
\medskip

В процессе работы над проектом были решены следующие задачи:

\begin{enumerate}
    \item Получение исходных текстов из Git-репозитория, эффективно используя особенности этой системы контроля версий;
    \item Получение семантической информации с использованием результатов работы любого корректного индексатора, поддерживающего формат LSIF;
    \item Хранение полученных данных в базе данных MongoDB, в том числе с использованием индексов для ускорения читающих запросов;
    \item Реализован веб-интерфейс на базе Visual Studio Code, использующий эти данные для отображения пользователю исходные текстов и предоставляющий расширенный функционал навигации;
    \item Простота развертывания достигается благодаря использованию файлов Dockerfile и docker-compose, описывающих процесс сборки и процедуру запуска системы;
\end{enumerate}

\medskip

В том числе существует несколько направлений дальнейшего развития этой работы:
\begin{enumerate}
    \item Разработка инструментария для автоматического запуска индексирования репозитория. На данный момент предлагается интеграция с используемыми в проекте системами CI/CD.
    \item Разработка модели разграничения доступов к отдельным файлам или директориям репозиториев.
    \item Улучшения интеграции интерфейса с системой контроля версий Git: поддержка функций «annotate» (кто последний менял ту или иную строку), просмотр списка коммитов и веток.
    \item Поддержка быстрого текстового поиска по репозиторию с использованием regexp.
\end{enumerate}

\clearpage