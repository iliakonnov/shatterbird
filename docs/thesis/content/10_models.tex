\addendum{Модели объектов}
\label{addendum:models}

Это приложение содержит описание моделей, используемых в проекте. Все модели описаны с помощью языка Typescript. Тип \texttt{Id<T>} обозначает идентифкатор объекта типа \texttt{T}.

\section{Модели для содержимого Git-репозитория}

Эти модели используются для хранения файлов и директорий Git-репозитория в базе данных:

\begin{itemize}
    \item Коллекция \texttt{commits}:
        \begin{minted}[gobble=8]{TypeScript}
        /**
         * Объект коммита, импортированного из Git-репозитория
         */
        export type Commit = { 
            /**
             * Идентификатор объекта в базе данных
             */
            _id: Id<Commit>, 
            
            /**
             * Хранит хэш, который используется для идентификации соответствующего объекта в Git
             */
            oid: string, 
            
            /**
             * Идентификатор корневой директории репозитория
             */
            root: Id<Node>, 
            
            /**
             * Список коммитов-родителей, если они также загружены в хранилище
             */
            parents: Array<Id<Commit>>,
        }
        \end{minted}

    \item Коллекция \texttt{nodes}:
        \begin{minted}[gobble=8]{TypeScript}
        /**
         * Объект в файловом дереве
         */
        export type Node = {
            /**
             * Идентификатор объекта в базе данных
             */
            _id: Id<Node>,
            
            /**
             * Хранит хэш, который используется для идентификации соответствующего объекта в Git
             */
            oid: string,
            
            /**
             * Содержимое объекта, в зависимости от его типа
             */
            content: {
                "Symlink": {
                    /**
                     * Путь, на который ссылается эта ссылка
                     */
                    target: string,
                }
            } | {
                "Directory": {
                    /**
                     * Объекты. входящие в эту директорию и их имена
                     */
                    children: { [key: string]: Id<Node> },
                }
            } | {
                "Text": {
                    /**
                     * Суммарный размер файла
                     */
                    size: bigint,
                    
                    /**
                     * Список строк, входящих в этот файл
                     */
                    lines: Array<Id<Line>>,
                }
            } | {
                "Blob": {
                    /**
                     * Суммарный размер файла
                     */
                    size: bigint,
                    
                    /**
                     * Идентификатор объекта, содержащего этот файл
                     */
                    content: Id<BlobFile>,
                }
            },
        };
        \end{minted}
        
    \item Коллекция \texttt{blobs}:
        \begin{minted}[gobble=8]{TypeScript}
        /**
         * Содержимое файла, который не удалось разделить на отдельные строки
         */
        export type BlobFile = {
            /**
             * Идентификатор объекта в базе данных
             */
            _id: Id<BlobFile>,
            
            /**
             * Содержимое файла (байты)
             */
            data: Array<number>,
        };
        \end{minted}
    
    \item Коллекция \texttt{lines}:
        \begin{minted}[gobble=8]{TypeScript}
        /**
         * Содержимое отдельной строки текстового файла
         */
        export type Line = {
            /**
             * Идентификатор объекта в базе данных
             */
            _id: Id<Line>,
            
            /**
             * Строка
             */
            text: string,
        };
        \end{minted}
\end{itemize}

\section{Модели для результатов индексирования}

Эти модели используются для хранения результатов работы LSIF в базе данных:

\begin{itemize}
    \item Коллекция «lines»:
        \begin{minted}[gobble=8]{TypeScript}
        /**
         * Описание подстроки в файле
         */
        export type Range = {
            /**
             * Идентификатор объекта в базе данных
             */
            _id: Id<Range>,
            
            /**
             * Идентификатор строки, в которой находится подстрока
             */
            line_id: Id<Line>,
            
            /**
             * Полный путь к этому файлу
             */
            path: Array<Id<Line>>,
            
            /**
             * Индекс первого символа подстроки
             */
            start: number,
            
            /**
             * Индекс конца подстроки
             */
            end: number,
        };
        \end{minted}
    
    \item Коллекция «edges»:
        \begin{minted}[gobble=8]{TypeScript}
        /**
         * Ребро графа, связывающее те или иные вершины
         */
        export type Edge = {
            /**
             * Идентификатор объекта в базе данных
             */
            _id: Id<Edge>,
            
            /**
             * Информация об ребре, предоставленная LSIF
             */
            data: {
                /**
                 * Вид ребра
                 */
                edge: "Contains"
                    | "Moniker"
                    | "NextMoniker"
                    | "Next"
                    | "PackageInformation"
                    | "Item"
                    | "Definition"
                    | "Declaration"
                    | "Hover"
                    | "References"
                    | "Implementation"
                    | "TypeDefinition"
                    | "FoldingRange"
                    | "DocumentLink"
                    | "DocumentSymbol"
                    | "Diagnostic",
                    
                /**
                 * Входящая вершина, если один
                 */
                in_v?: Id<Vertex>,
                
                /**
                 * Входящие вершины, если несколько
                 */
                in_vs?: Array<Id<Vertex>>,
                
                /**
                 * Исходящая вершина
                 */
                out_v: Id<Vertex>,
            },
        };
        \end{minted}
        
    \item Коллекция «vertices»:
        \begin{minted}[gobble=8]{TypeScript}
        /**
         * Вершина графа
         */
        export type Vertex = {
            /**
             * Идентификатор объекта в базе данных
             */
            _id: Id<Vertex>,
            
            /**
             * Информация об вершине, предоставленная LSIF
             */
            data: {
                /**
                 * Вид вершины
                 */
                vertex: "MetaData"
                      | "Project"
                      | "Document"
                      | "Range"
                      | "ResultSet"
                      | "Moniker"
                      | "PackageInformation"
                      | "DefinitionResult"
                      | "DeclarationResult"
                      | "TypeDefinitionResult"
                      | "ReferenceResult"
                      | "ImplementationResult"
                      | "FoldingRangeResult"
                      | "HoverResult"
                      | "DocumentSymbolResult"
                      | "DocumentLinkResult"
                      | "DiagnosticResult",
                
                // Также содержит дополнительные поля, зависящие от конкретного вида вершины.
            },
        };
        \end{minted}
\end{itemize}

\section{Объект FullNode}
\label{models:fullnode}

Этот объект не хранится в базе данных, но используется при взаимодействии с клиентской частью.

\begin{minted}{TypeScript}
/**
 * Хранит полную информацию об узле файлового дерева
 */
export type FullNode = {
    /**
     * Идентифкатор этого узла в базе данных
     */
    _id: Id<Node>,

    /**
     * Тип узла
     */
    kind: "Symlink" | "Directory" | "Text" | "Blob",

    content: {
        "Symlink": {
            /**
             * Ссылка на целевой файл
             */
            target: string,
        }
    } | {
        "Directory": {
            /**
             * Дочерние узлы этой директории вместе с их именами
             */
            children: {
                [key: string]: {
                    /**
                     * Идентифкатор этого узла в базе данных
                     */
                    _id: Id<Node>,
                    
                    /**
                     * Тип узла
                     */
                    kind: "Symlink" | "Directory" | "Text" | "Blob",
                }
            },
        }
    } | {
        "Text": {
            /**
             * Размер файла в байтах
             */
            size: bigint,
            /**
             * Строки файла
             */
            lines: Array<{
                /**
                 * Идентификатор объекта в базе данных
                 */
                _id: Id<Line>,
                
                /**
                 * Текст строки
                 */
                text: string,
            }>,
        }
    } | {
        "Blob": {
            /**
             * Размер файла в байтах
             */
            size: bigint,
            
            /**
             * Идентификатор этого файла в базе данных
             */
            content: Id<BlobFile>,
        }
    },
};
\end{minted}


\clearpage