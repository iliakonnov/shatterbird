\usepackage[nonumberlist,nogroupskip,xindy={glsnumbers=false,codepage=utf8,language=russian}]{glossaries}
\usepackage[automake=immediate]{glossaries-extra}
% \setabbreviationstyle[acronym]{short-footnote}
\setglossarystyle{list}
\renewcommand{\glossarysection}[2][]{}
\renewcommand{\glsnamefont}[1]{\textrm{#1}}
\makeglossaries

\newglossaryentry{IDE}
{
    name=IDE,
    description={(Integrated Development Environment) — интегрированная среда разработки, предоставляющая разработчику большое количество функционала для работы с кодом проектов}
}

\newglossaryentry{LSP}
{
    name=LSP,
    description={(Language Server Protocol) \cite{lsp-book} — разработанный Microsoft протокол взаимодействия редактора кода и специального сервиса, который способен анализировать код проекта и предоставлять редактору продвинутый функционал, сравнимый с классическими IDE}
}

\newglossaryentry{LSIF}
{
    name=LSIF,
    description={(Language Server Index Format) \cite{lsif} — специальный формат, в котором LSP-сервер может вывести информацию о проекте таким образом, что её можно переиспользовать в дальнейшем, не имея при этом запущенного экземпляра сервера и даже кода проекта}
}

\newglossaryentry{symbol}
{
    name={Символ},
    text={символ},
    description={(в контексте исходного кода) — идентифкатор в тексте программе, однажды где-либо определенный. Например, в строке «\texttt{def func():}» символом будет являться «\texttt{func}». Также символами являются имена переменных, классов и многое другое}
}

\newglossaryentry{repo}
{
    name={Репозиторий},
    text={репозиторий},
    description={— файлы какого-либо программного проекта, включающие, среди прочего, весь его исходный код. В данной работе под репозиторием имеется ввиду Git-репозиторий, в котором все файлы также находятся под управлением системы контроля версий Git, которая позволяет эффективно создавать новые версии репозитория и делиться внесенными изменениями с другими разработчиками}
}