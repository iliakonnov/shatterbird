% Языки, которые могут быть использованы.
\usepackage[russian,english]{babel}

\usepackage{fontspec}
\setmainfont[Ligatures=TeX]{Times New Roman}
\setmonofont[Mapping=,Ligatures=]{Courier New}
\addtokomafont{disposition}{\rmfamily}

% Чтобы сделать буковки ещё красивее.
\usepackage{microtype}

% Картинки, вращение и всё такое. Причём даже в режиме черновика
\usepackage[final]{graphicx}
\usepackage{float}
\usepackage[draft=no]{svg}
\usepackage{pdfpages}

\usepackage{tikz}
\usetikzlibrary{fit,arrows.meta,decorations.pathmorphing,backgrounds}

% Чтобы легко менять стиль списков
\usepackage[shortlabels]{enumitem}

\setmainfont{Times New Roman} % Шрифт для основного текста
\setmonofont{Courier New}     % Шрифт для моноширинного текста
\usepackage[
    left=30mm,       % левое поле: 3см
    right=15mm,      % правое поле: 1.5см
    top=20mm,        % верхнее поле: 2см
    bottom=20mm,     % нижнее поле: 2см
    includeheadfoot,
    footskip=12.5mm,
    headheight=15mm,
]{geometry}
\usepackage[
    indent=12.5mm  % абзацный отступ: 1.25см
    skip=10pt plus1pt,
]{parskip}
\onehalfspacing\selectfont  % полуторный интервал

\usepackage{unicode-math}
\usepackage[final]{listings}
\usepackage[autostyle]{csquotes}
\usepackage{adjustbox}
\usepackage{relsize}
\usepackage{upquote}

\MakeAutoQuote{«}{»}

\usepackage{minted}
\definecolor{backcolour}{rgb}{0.95,0.95,0.92}
\setminted{
    linenos,
    mathescape,
    breaklines,
    fontsize=\small,
    bgcolor=backcolour,
    xleftmargin=0pt,
    numbersep=8pt,
}

% https://qna.habr.com/answer?answer_id=1140867
\bibliographystyle{gost2008}
\usepackage[
	style=gost-numeric, % стиль цитирования и библиографии, см. документацию biblatex-gost
	language=autobib,
	autolang=other, % многоязычная библиография
	parentracker=true,
	backend=biber,
	hyperref=true,
	bibencoding=utf8,
	defernumbers=true,
]{biblatex}
\addbibresource{bibliography.bib}
\DeclareSourcemap{ %модификация bib файла перед тем, как им займётся biblatex
    \maps{
        \map{% перекидываем значения полей language в поля langid, которыми пользуется biblatex
            \step[fieldsource=language, fieldset=langid, origfieldval, final]
            \step[fieldset=language, null]
        }
        \map{% перекидываем значения полей numpages в поля pagetotal, которыми пользуется biblatex
            \step[fieldsource=numpages, fieldset=pagetotal, origfieldval, final]
            \step[fieldset=numpages, null]
        }
        \map{% перекидываем значения полей pagestotal в поля pagetotal, которыми пользуется biblatex
            \step[fieldsource=pagestotal, fieldset=pagetotal, origfieldval, final]
            \step[fieldset=pagestotal, null]
        }
        \map[overwrite]{% перекидываем значения полей shortjournal, если они есть, в поля journal, которыми пользуется biblatex
            \step[fieldsource=shortjournal, final]
            \step[fieldset=journal, origfieldval]
            \step[fieldset=shortjournal, null]
        }
        \map[overwrite]{% перекидываем значения полей shortbooktitle, если они есть, в поля booktitle, которыми пользуется biblatex
            \step[fieldsource=shortbooktitle, final]
            \step[fieldset=booktitle, origfieldval]
            \step[fieldset=shortbooktitle, null]
        }
        \map{% если в поле medium написано "Электронный ресурс", то устанавливаем поле media, которым пользуется biblatex, в значение eresource.
            \step[fieldsource=medium,
            match=\regexp{Электронный\s+ресурс},
            final]
            \step[fieldset=media, fieldvalue=eresource]
            \step[fieldset=medium, null]
        }
        \map[overwrite]{% стираем значения всех полей issn
            \step[fieldset=issn, null]
        }
        \map[overwrite]{% стираем значения всех полей abstract, поскольку ими не пользуемся, а там бывают "неприятные" латеху символы
            \step[fieldsource=abstract]
            \step[fieldset=abstract,null]
        }
        \map[overwrite]{ % переделка формата записи даты
            \step[fieldsource=urldate,
            match=\regexp{([0-9]{2})\.([0-9]{2})\.([0-9]{4})},
            replace={$3-$2-$1$4}, % $4 вставлен исключительно ради нормальной работы программ подсветки синтаксиса, которые некорректно обрабатывают $ в таких конструкциях
            final]
        }
        \map[overwrite]{ % стираем ключевые слова
            \step[fieldsource=keywords]
            \step[fieldset=keywords,null]
        }
        \map[overwrite]{ % записываем информацию о типе публикации в ключевые слова
            \step[fieldsource=authorvak,final=true]
            \step[fieldset=keywords,fieldvalue={,biblioauthorvak},append=true]
        }
        \map[overwrite]{ % записываем информацию о типе публикации в ключевые слова
            \step[fieldsource=authorscopus,final=true]
            \step[fieldset=keywords,fieldvalue={,biblioauthorscopus},append=true]
        }
        \map[overwrite]{ % записываем информацию о типе публикации в ключевые слова
            \step[fieldsource=authorwos,final=true]
            \step[fieldset=keywords,fieldvalue={,biblioauthorwos},append=true]
        }
        \map[overwrite]{ % записываем информацию о типе публикации в ключевые слова
            \step[fieldsource=authorconf,final=true]
            \step[fieldset=keywords,fieldvalue={,biblioauthorconf},append=true]
        }
        \map[overwrite]{ % записываем информацию о типе публикации в ключевые слова
            \step[fieldsource=authorother,final=true]
            \step[fieldset=keywords,fieldvalue={,biblioauthorother},append=true]
        }
        \map[overwrite]{ % записываем информацию о типе публикации в ключевые слова
            \step[fieldsource=authorpatent,final=true]
            \step[fieldset=keywords,fieldvalue={,biblioauthorpatent},append=true]
        }
        \map[overwrite]{ % записываем информацию о типе публикации в ключевые слова
            \step[fieldsource=authorprogram,final=true]
            \step[fieldset=keywords,fieldvalue={,biblioauthorprogram},append=true]
        }
        \map[overwrite]{ % добавляем ключевые слова, чтобы различать источники
            \perdatasource{biblio/external.bib}
            \step[fieldset=keywords, fieldvalue={,biblioexternal},append=true]
        }
        \map[overwrite]{ % добавляем ключевые слова, чтобы различать источники
            \perdatasource{biblio/author.bib}
            \step[fieldset=keywords, fieldvalue={,biblioauthor},append=true]
        }
        \map[overwrite]{ % добавляем ключевые слова, чтобы различать источники
            \perdatasource{biblio/registered.bib}
            \step[fieldset=keywords, fieldvalue={,biblioregistered},append=true]
        }
        \map[overwrite]{ % добавляем ключевые слова, чтобы различать источники
            \step[fieldset=keywords, fieldvalue={,bibliofull},append=true]
        }
%        \map[overwrite]{% стираем значения всех полей series
%            \step[fieldset=series, null]
%        }
        \map[overwrite]{% перекидываем значения полей howpublished в поля organization для типа online
            \step[typesource=online, typetarget=online, final]
            \step[fieldsource=howpublished, fieldset=organization, origfieldval]
            \step[fieldset=howpublished, null]
        }
    }
}

\newcommand*{\fullref}[1]{\hyperref[{#1}]{\ref*{#1} (\nameref*{#1})}}

\usepackage[
    % showframe,
    includeheadfoot,
	left=30mm,
	right=15mm,
	top=20mm,
	bottom=20mm,
	footskip=12.5mm,
	headheight=15mm,
]{geometry}
\savegeometry{original}

\usepackage{setspace}
\usepackage[skip=10pt plus1pt, indent=12.5mm]{parskip}
\setlist[itemize]{leftmargin=12.5mm}
\setlist[enumerate]{leftmargin=12.5mm}
\onehalfspacing\selectfont

\geometry{ignoreheadfoot}
\savegeometry{nofooter}

\loadgeometry{original}

%% Некоторые переменные, которые появляются более одного раза

% Название курсовой. Не забывать проценты в конце строк.
\newcommand{\docTitle}{%
	Сервис для индексирования и просмотра репозиториев%
}

\newcommand{\docTitleEng}{%
	Repository Indexing and Viewing Service%
}

% То же самое название, но в родительном падеже
\newcommand{\docTitleGenitive}{%
	Сервиса для индексирования и просмотра репозиториев%
}

% Год написания курсовой.
\newcommand{\YEAR}{
	\the\year{}
}

% RU — код страны
% 17701729 — код НИУ ВШЭ
% 05.15 — регистрационный номер (https://docs.cntd.ru/document/566085681)
% 		В данном случае, Прикладное ПО / Информационные системы для решения специфических отраслевых задач
% 01 — номер редакции документа
% ТЗ — код вида документа
% 01 — номер документа данного вида
%  1 — номер части документа
\newcommand{\docId}{RU.17701729.03.10-01 ТЗ 01-1}


\usepackage{fancyhdr}
\pagestyle{fancy}
\fancyhf{}
\chead{}
\fancyhead[RO]{%
    % В правый верхний колотитул пихаю номер приложения
    {\ifnum\value{addendum}>0 ПРИЛОЖЕНИЕ \Asbuk{addendum} \fi}
}
\cfoot{\thepage}

\renewcommand{\headrulewidth}{0pt}
\renewcommand{\footrulewidth}{0pt}

% Повернуть на 90 градусов
\newcommand{\rot}[2]{\rotatebox[origin=c]{90}{\enspace\parbox{#1 - 0.5em}{#2}}}

% Повторить #1 раз текст #2: \Repeat{#1}{#2}
\usepackage{expl3}
\ExplSyntaxOn
\cs_new_eq:NN \Repeat \prg_replicate:nn
\ExplSyntaxOff

% Продвинутые таблицы
\usepackage{tabularx}
\usepackage{multirow}
\usepackage{xltabular}
% И сразу делаем чтобы колонки X центрировались по вертикали
\def\tabularxcolumn#1{m{#1}}
\newcolumntype{Y}{>{\centering\arraybackslash}X}

% Расчет всяких размеров
\usepackage{calc}

% Названия глав:
\RedeclareSectionCommand[
  beforeskip=0pt,
  font=\Large,
  afterskip=12pt]{chapter}
\RedeclareSectionCommand[
  beforeskip=8pt,
  font=\large,
  afterskip=4pt]{section}

\let\oldaddsec\addsec
\renewcommand{\addsec}[1]{\oldaddsec[#1]{\hfill{#1}\hfill\mbox{}}}
\let\oldaddchap\addchap
\renewcommand{\addchap}[1]{\oldaddchap[#1]{\hfill{#1}\hfill\mbox{}}}

\renewcommand*\chapterpagestyle{fancy}

% После названия раздела надо делать отступы у абзацев.
\usepackage{indentfirst}

% https://tex.stackexchange.com/a/347803
\usepackage{varwidth}

\usepackage[perpage, symbol*]{footmisc}

\newcommand{\placename}{
    \begin{minipage}[b][1cm][b]{3cm}
	    \underline{\hspace{3cm}}
    \end{minipage}
}
\newcommand{\placedate}{
	«\underline{\hspace{1em}}» \underline{\hspace{3cm}} \YEAR г.
}
% \newcommand{\facebook}{\footnote{Meta Platforms, а также принадлежащие ей социальная сеть Facebook — признана экстремистской организацией, её деятельность в России запрещена}}
\newcommand{\facebook}{}

\newcommand{\textunderset}[2]{\begin{tabular}[t]{@{}c@{}}#2\\[-0.3em]\scriptsize#1\end{tabular}}
\newcommand{\textoverset}[2]{\begin{tabular}[b]{@{}c@{}}\scriptsize#1\\[-0.3em]#2\end{tabular}}
\newcommand{\placeholder}[1]{\textunderset{#1}{\underline{\hspace{3cm}}}}

% Подсчет числа страниц
\usepackage{lastpage}

\usepackage[figure,page]{totalcount}
\usepackage{totcount}
\newtotcounter{citenum}
\newtotcounter{chaps}
\AtEveryBibitem{\stepcounter{citenum}}

\usepackage[toc,title,titletoc,header]{appendix}
\renewcommand{\appendixname}{{Приложение}}
\renewcommand{\appendixtocname}{{Приложения}}

\AddToHook{cmd/appendix/after}{%
  \AddToHook{cmd/chapterformat/before}{\chapapp\nobreakspace}%
  \AddToHook{cmd/chaptermarkformat/before}{\chapapp\nobreakspace}%
  \addtocontents{toc}{\appendixtocentry}%
}

\NewDocumentCommand{\appendixtocentry}{}{%
  \DeclareTOCStyleEntry[
    entrynumberformat=\tocappendixnumber,
    dynnumwidth
  ]{chapter}{chapter}%
}
\NewDocumentCommand{\tocappendixnumber}{m}{%
  \appendixname~#1%
}

\newtotcounter{addendum}
\newcommand{\addendum}[1]{
    \stepcounter{addendum}
	
	% Сбросить все счетчики
	\setcounter{equation}{0}
    \setcounter{figure}{0}
    \setcounter{table}{0}

	\chapter{#1}%
}

% Ещё больше подсекций
\newcommand{\subsubsubsection}[1]{\paragraph{#1}\mbox{}\\}
\setcounter{secnumdepth}{4}
\setcounter{tocdepth}{4}

% TODOшечки
\usepackage[obeyDraft,textsize=footnotesize,textwidth=35mm]{todonotes}
\presetkeys%
    {todonotes}%
    {inline,backgroundcolor=yellow}{}

% Добавляем гипертекстовое оглавление в PDF
% hyperref должен быть последним
\usepackage[hidelinks,draft=false]{hyperref}