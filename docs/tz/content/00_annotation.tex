\section*{АННОТАЦИЯ}

Техническое задание – это основной документ, оговаривающий набор требований и
порядок создания программного продукта, в соответствии с которым производится разработка
программы, ее тестирование и приемка.

Настоящее Техническое задание на разработку «\docTitleGenitive» содержит следующие разделы: «Введение», «Основание для разработки», «Назначение разработки», «Требования к программе», «Требования к программным документам», «Технико-экономические показатели», «Стадии и этапы разработки» и «Порядок контроля и приёмки».

В разделе «Введение» указано наименование и краткая характеристика области применения «\docTitleGenitive».

В разделе «Основания для разработки» указан документ на основании, которого ведётся разработка и наименование темы разработки.

В разделе «Назначение разработки» указано функциональное и эксплуатационное назначение программного продукта.

Раздел «Требования к программе» содержит основные требования к функциональным характеристикам, к интерфейсу, к надёжности, к условиям эксплуатации, к составу и параметрам технических средств, к информационной и программной совместимости, к маркировке и упаковке, к транспортировке и хранению, а также специальные требования.

Раздел «Требования к программным документам» содержит предварительный состав программной документации и специальные требования к ней.

Раздел «Технико-экономические показатели» содержит ориентировочную экономическую эффективность, предполагаемую годовую потребность, экономические преимущества разработки «\docTitleGenitive».

Раздел «Стадии и этапы разработки» содержит стадии разработки, этапы и содержание работ, сроки выполнения.

В разделе «Порядок контроля и приемки» указаны общие требования к приемке работы.

Настоящий документ разработан в соответствии с требованиями:
\begin{enumerate}[1)]
	\item ГОСТ 19.101-77 Виды программ и программных документов \cite{gost:19.101-77};
	\item ГОСТ 19.102-77 Стадии разработки \cite{gost:19.102-77};
	\item ГОСТ 19.103-77 Обозначения программ и программных документов \cite{gost:19.103-77};
	\item ГОСТ 19.104-78 Основные надписи \cite{gost:19.104-78};
	\item ГОСТ 19.105-78 Общие требования к программным документам \cite{gost:19.105-78};
	\item ГОСТ 19.106-78 Требования к программным документам, выполненным печатным способом \cite{gost:19.106-78};
	\item ГОСТ 19.201-78 Техническое задание. Требования к содержанию и оформлению \cite{gost:19.201-78};
	\item Изменения к данному Техническому заданию оформляются согласно ГОСТ 19.603-78 \cite{gost:19.603-78}, ГОСТ 19.604-78 \cite{gost:19.604-78}.
\end{enumerate}

\clearpage