\section{ТРЕБОВАНИЯ К ПРОГРАММЕ}
\label{requirements}

\subsection{Требования к функциональным характеристикам}
\label{requirements.features}
    \subsubsection{Требования к составу выполняемых функций}
        Система должна предоставлять следующий функционал:
        \begin{enumerate}[series=requirements]
        	\item обновление ранее построенных индексов для новых версий репозитория;
        	\item поддержка индексирования языков Rust и TypeScript;
        	\item предоставлять возможность найти все использования в кодовой базе для типов, методов и переменных;
        	\item предоставлять возможность перейти к определению того или иного идентификатора;
        	\item предоставлять web-интерфейс, через который можно воспользоваться вышеописанным функционалом;
        \end{enumerate}

    \subsubsection{Требования к временным характеристикам}
        Временные характеристики зависят от характеристик устройства, на котором производятся вычисления, а также от свойств проекта (связность модулей, количество зависимостей, объём кодовой базы).
        
        Время индексирования должно, в первую очередь, зависеть от объёма очередных изменений и их транзитивных последствий. Оно не должно значительно зависеть от объёма незатронутой кодовой базы и от количества ранее построенных индексов.

        Время ответов на поисковые запросы не должно превышать двух секунд. При этом допускается отдавать неполный ответ, если количество найденных строк оказывается слишком большим.

    \subsubsection{Требования к организации входных данных}
    	Входными данными для индексирования являются:
    	\begin{itemize}
    	    \item Git-репозиторий с исходными текстами (в виде файлов на диске компьютера);
    	    \item Идентификатор очередного коммита в репозитории (при обновлении уже построенного индекса);
    	    \item Результат работы внешней программы-индексатора, при необходимости;
    	\end{itemize}
    	
    	Входными данными для запросов вида «перейти к определению» или «найти использования» того или иного идентификатора:
    	\begin{itemize}
    	    \item тип запроса: «перейти к определению» или «найти использования»;
    	    \item положение искомого идентификатора: идентификатор коммита, путь к файлу, номер строки, номер колонки в ней;
    	\end{itemize}
    
    \subsubsection{Требования к организации выходных данных}
    	Требования к формату построенного индекса не предъявляются.
    	
    	Для поисковых запросов выходными данными является набор вхождений, где каждое вхождение представлено следующими полями:
    	\begin{itemize}
    	    \item идентификатор коммита
    	    \item путь к файлу
    	    \item номер строки
    	    \item номер колонки в ней
    	    \item длина вхождения
    	\end{itemize}

\subsection{Требования к надёжности}
\label{requirements.quality}
    \subsubsection{Требования к обеспечению надёжного (устойчивого) функционирования программы}
    	Сервис должен адекватно работать в случае, если некоторые индексы по той или иной причине отсутствуют. Возможно, с потерей части функционала и выдачей неполных ответов.
    	
    	В случае неожиданной ошибки при построении тех или иных индексов, должны производиться повторные попытки (с ограничением на число попыток).
    	
    	Операции, требующие слишком большое количество ресурсов, должны отменяться с соответствующим сообщением об ошибке, возможно, с возвратом частичного ответа. «Тяжелые» запросы не должны приводить к отказу.

    \subsubsection{Отказы из-за некорректных действий оператора}
    	В случае передачи некорректных запросов, система должна адекватно на них реагировать, отдавая сообщение о некорректности запроса.

% \subsection{Условия эксплуатации}
% 	

\subsection{Требования к составу и параметрам технических средств}
    \subsubsection{Требования для запуска серверной части}
        \begin{itemize}
            \item Компьютер, исполняющий операционную систему Linux с установленным и запущенным сервисом Docker.
            \item 1 ГиБ свободной оперативной памяти.
            \item 5 ГиБ свободного места на диске.
        \end{itemize}
        
        В зависимости от размера загруженных в базу репозиториев и индексов, требования могут варьироваться в большую сторону.
    
   \subsubsection{Требования для использования пользовательской части}
        Требуется веб-бразуер, соответствующий по возможностям браузеру Firefox 125, а также сетевой доступ к запущенной серверной части системы. Требования для использования соответствуют требования используемого веб-браузера.

\subsection{Требования к информационной и программной совместимости}
    \subsubsection{Требования к информационным структурам и методам решения}
    	Требования к программным средствам не предъявляются.

    \subsubsection{Требования к программным средствам, используемыми программой}
    	Требования к программным средствам не предъявляются.
    
    \subsubsection{Требования к исходным кодам и языкам программирования}
    	Процесс сборки (компиляции) исполняемых файлов должен быть подробно задокументирован и полностью автоматизирован таким образом, чтобы было возможно получить исполняемые файлы без внесения каких-либо дополнительных изменений в файлы. В качестве системы контроля версий для исходных кодов необходимо использовать Git \cite{git}, причём репозиторий с историей считается неотъемлемой частью исходных кодов и должен быть передан вместе с текстами программы.

    \subsubsection{Требования к защите информации и программы}
    	Требования к защите информации и программы не предъявляются.

\subsection{Требования к маркировке и упаковке}
    Программное изделие должно иметь маркировку с обозначением наименования
    изделия, темы разработки, фамилии, имени и отчества исполнителя и
    руководителя разработки, учебной группы студента и года выпуска изделия.

\subsection{Требования к транспортировке и хранению}
    Специальные требования к транспортировке не предъявляются.

\subsection{Специальные требования}
    Специальные требования к данной программе не предъявляются.

\clearpage